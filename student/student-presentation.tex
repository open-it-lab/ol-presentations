%\documentclass[xetex,mathserif,serif]{beamer}
\documentclass{beamer}
\renewcommand{\tiny}{\fontsize{4}{14}\selectfont}

\usepackage{hyperref}
\usepackage{fontspec} 
\usepackage{xunicode} %Unicode extras!
\usepackage{xltxtra} %Fixes 
\usefonttheme{professionalfonts}
\setmainfont{Linux Libertine}
%\setmonofont[Scale=0.86]{DejaVu Sans Mono}
\setmonofont{Liberation Mono}
%\setromanfont{Silkscreen}
%\setsansfont{DejaVu Sans}
\setsansfont{Droid Sans}

\usepackage[final,expansion=true,protrusion=true,spacing=true,kerning=true]{microtype}
\usetheme{openlab} 
\setbeamertemplate{navigation symbols}{}
\usepackage{graphicx}

\title[Students]{Open IT Lab} 
\author{Jarrell Waggoner} 
\institute[Open IT Lab] {Open IT Lab\\
  \medskip
      {\emph{waggonej@email.sc.edu}} }
\date{\today}

\usebackgroundtemplate{\includegraphics[width=\paperwidth]{img/bg.png}}

\begin{document}
\rm

{
  \usebackgroundtemplate{\includegraphics[width=\paperwidth]{img/bg-title.png}} 
  \begin{frame}
%    \titlepage
    \vspace{18em}

    \begin{center}\large{\textcolor{beamer@mygrey}{Jarrell Waggoner}}\end{center}

%    \begin{center}\small{\textcolor{beamer@mygreen}{waggonej@email.sc.edu}}\end{center}

%    \begin{center}\small{\textcolor{beamer@mygrey}{\today}}\end{center}
  \end{frame}
}

\begin{frame}
  \frametitle{Introduction}
  \begin{center}\begin{LARGE}What does it mean to be Open?\end{LARGE}\end{center}
\end{frame}

\begin{frame}
  \frametitle{What does the dictionary say?}
  \begin{center}\end{center}
\end{frame}

\begin{frame}
  \frametitle{Our Definition}
  \begin{center}
    \begin{LARGE}
      The freedom to share, explore, reproduce, and contribute ideas, and projects based on these ideas
    \end{LARGE}
\end{center}
\end{frame}

\begin{frame}
  \frametitle{What can be open?}
  \begin{columns}
    \column{0.45\textwidth}
    \begin{center}
      \begin{LARGE}Software\end{LARGE}
      
      \vspace{5em}

      \begin{LARGE}Hardware\end{LARGE}

      \vspace{5em}

      \begin{LARGE}Content\end{LARGE}
    \end{center}
    \column{0.45\textwidth}
    \begin{center}
      \includegraphics[width=0.4\textwidth]{img/opensource.png}

      \vspace{1em}

      \includegraphics[width=0.4\textwidth]{img/opensourcehardware.png}

      \vspace{1em}

      \includegraphics[width=0.8\textwidth]{img/cc.png}
    \end{center}
  \end{columns}

\end{frame}

\begin{frame}
  \frametitle{Formal Definition}
  \begin{center}
    \begin{block}{Software}
      Allows free distribution, modification the source code, allows
      derived works, does not discriminate or have limiting license
      restrictions\textemdash\textcolor{beamer@myblue}{\href{http://opensource.org/docs/osd}{opensource.org}}
    \end{block}

    \begin{block}{Hardware}
      hardware whose design is made publicly available so that anyone
      can study, modify, distribute, make, and sell the design or
      hardware based on that design\textemdash\textcolor{beamer@myblue}{\href{http://freedomdefined.org/OSHW}{OSHW}}
    \end{block}

    \begin{block}{Content}
      A piece of content or data is open if anyone is free to use,
      reuse, and redistribute it--subject only, at most, to the
      requirement to attribute and
      share-alike\textemdash\textcolor{beamer@myblue}{\href{http://www.opendefinition.org/}{opendefinition.org}}

      \vspace{1em}
      
      Open content encourages the 4Rs: Reuse, Revise, Remix, Redistribute
      \href{http://www.opencontent.org/definition/}{opencontent.org/definition}
    \end{block}
  \end{center}
\end{frame}

%% \begin{frame}
%%   \frametitle{Slide 2 Title}

%% \end{frame}

\end{document}
